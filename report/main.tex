\documentclass[12pt]{article}
\usepackage[utf8]{inputenc}
\usepackage{listings}
\usepackage{amsfonts}
\usepackage{indentfirst}
\usepackage{hyperref}
\usepackage[T1]{fontenc}
\usepackage{amsmath}
\usepackage{amssymb}
\usepackage[makeroom]{cancel}
\usepackage{amsthm}

\title{COMP 424 Final Project Report \\ Colosseum Surivival}
\author{Taha Rhaouti - $260974004$ \\ Reda Toumi - $260922679$}
\date{December 2023}

\newtheorem{definition}{Definition}[section]
\newtheorem{theorem}{Theorem}[section]
\newtheorem{corollary}{Corollary}[theorem]
\newtheorem{lemma}[theorem]{Lemma}

\begin{document}

\maketitle

\section{Introduction}

\pagebreak
\section*{Bibliography}

\begin{flushleft}
    "A Brief History of $\pi$." n.d. \href{https://www.exploratorium.edu/sites/default/files/pdfs/history_of_pi.pdf}{\underline{link}}
\end{flushleft}

\begin{flushleft}
    Charles Hermite; letter to C.W. Borchardt, "Men of Mathematics", E. T. Bell, New York 1937, p. 464.
\end{flushleft}

\begin{flushleft}
    Simmons, George F.. Calculus Gems: Brief Lives and Memorable Mathematics. United States, American Mathematical Society, 2020.
\end{flushleft}

\begin{flushleft}
    Vo, Thanh, and Huan. n.d. “INTRODUCTION to TRANSCENDENTAL NUMBERS.” Accessed April 9, 2023. \href{http://www.math.toronto.edu/vohuan/Notes/mas216_report.pdf}{\underline{link}}
\end{flushleft}

\begin{flushleft}
    Wikipedia Contributors. 2023. “Liouville Number.” Wikipedia. Wikimedia Foundation. April 10, 2023. \href{https://en.wikipedia.org/wiki/Liouville_number#:~:text=Liouville%20numbers%20are%20%22almost%20rational,algebraic%20irrational%20number%20can%20be}{\underline{link}}
\end{flushleft}

\begin{flushleft}
    Ross, Marty. 2018. \href{https://www.qedcat.com/notes/e%20%2B%20pi%20transcendental.pdf}{\underline{link}}
\end{flushleft}

\begin{flushleft}
    Michael Spivak. Calculus. Cambridge university press, third edition,2006
\end{flushleft}

\begin{flushleft}
    Wikipedia Contributors. 2023. “Lindemann–Weierstrass Theorem.” Wikipedia. Wikimedia Foundation. February 28, 2023. \href{https://en.wikipedia.org/wiki/Lindemann%E2%80%93Weierstrass_theorem}{\underline{link}}
\end{flushleft}

\begin{flushleft}
    Paul Richard Halmos, John Ewing, F.W. Gehring (1991). “PAUL HALMOS Celebrating 50 Years of Mathematics: Celebrating 50 Years of Mathematics”, p.21, Springer Science \& Business Media
\end{flushleft}

\end{document}